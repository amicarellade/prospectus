\documentclass{proc}

\begin{document}

\title{Visualizing NFL Draft Data}

\author{Dante Amicarella, Jack Lafond, Mason Perham}

\maketitle

\section{Introduction}
The NFL is a multi billion dollar industry whose fans, media, and investors continue to grow each year both in the US and globally. Each off-season much of the excitement and anticipation continues as fans and analysts await the next NFL draft. For many teams and fanbases the draft can either mean years of glory to come, something that the Chiefs and Patrick Mahomes know all too well. However, for others this could mean wasted picks and years of sub-par seasons. 
As of now there is no formal way of viewing NFL draft data for analysts or NFL fanatics around the world. The data from these drafts is greatly interconnected with much of the overall American Football data landscape, and as a result can be complicated to properly view. Each player picked represents an athlete with physical scores and measurements, a college that helped develop that athlete, a position that represents that athlete's particular set of skills, a round and pick number that represents the assessed value of that player by the team that picked them, and a possible future of success and accolades for that team. 
In this project we will explore various visualizations and attempt to encode and display draft data in a way that makes extracting information from this dataset easier for fans and draft analysts. We have gathered a large sample of NFL draft data that spans from 2010-2020. In this data set we include all relevant variables for the draft like player position, college, draft pick, and the team that drafted them. We have also gathered career accolades for each of these players like number of MVP awards, Super Bowl wins, and other NFL honors. Due to the complex nature of the dataset we will utilize different types of visualizations like network and flow visualizations, such as Sankey and Hive plots, as well as dimension comparison plots like spider charts and parallel coordinate plots. 
Through this project we hope to establish efficient ways of visualizing the NFL draft so that we can improve the current state of NFL draft visualizations for fans, analysts, and media.
\section{Project Description}
This project aims to create functional visualizations for NFL Draft data. 
\section{Project Type}
This project will be a web-application that allows the user to view NFL data in two different tabs. One tab will present data in a coordinated visualization, while the other will be in a scrolley-telling format. 
\section{Audience} 
If we are to be successful in establishing efficient ways of visualizing the NFL draft then this project will serve to help NFL analysts who wish to improve their current methods for visualizing the draft. This will also help media companies who cover the NFL draft as they will be able to provide better visualization techniques for their followers to consume. Using valid visualizations will also improve the quality of the overall story that we discover, which will serve to entertain and possibly educate NFL fans. 
	If we are unsuccessful in improving NFL draft visualization then this could also serve NFL analysts and media companies. In this case we would show which visualizations and techniques are not suited for the NFL draft. This would help save time for analysts and media companies also exploring this space, and could provide new research interests to explore for them.
\begin{quote}
\textit{Who is the audience for this project? 
How does it meet their needs? 
What happens if their needs remain unmet?}
\end{quote}

\section{Approach}
\subsection{Details}
\begin{quote}
\textit{What is your approach?}

\end{quote}

\subsection{Evidence for Success}
\begin{quote}
\textit{Why do you think it will work?} 
\end{quote}


\section{Best-case Impact Statement}
\begin{quote}
\textit{In the best-case scenario, what would be the impact statement (conclusion statement) for this project? \cite{wijk2005value}}
\end{quote}

\section{Major Milestones}

\section{Obstacles}

\subsection{Major obstacles} % (if these fail, the project is over)

\subsection{Minor obstacles}
\begin{itemize}
    \item Having no experience with scrolley-telling formats.
\end{itemize}
\section{Resources Needed}
\begin{quote}
We will not need any additional resources for this project.
\end{quote}

\section{5 Related Publications}
\begin{quote}
\textit{List 5 major publications that are most relevant to this project, and how they are related (sample citation \cite{wijk2005value}).}
\begin{itemize}
    \item \cite{perin2018state} This publication speaks about the current state of the art sports data visualization. Our project is about NFL draft data.
    \item \cite{krzywinski2012hive} This publication speaks about the use of hive plots in data visualization, and we are using a hive plot as our main visualization.
    \item \cite{janetzko2016enhancing} This publication is about enhancing parallel coordinate plots, specifically, introducing different distribution plots on the  axes to better display the information in the data. We will be using PCP's in our project and we will also need to figure out how to convey information when we have many player data points converging to one area.
    \item \cite{morth2023SCrollyVis}https://ieeexplore.ieee.org/abstract/document/9887905 - Scrollytelling - Writing and Design
    \item \cite{morth2023scrollyvis} This article demonstrates the ability of ScrollyVisualizations 
\end{itemize}
\end{quote}

\section{Define Success}
This project will be publishable if we are able to create all the visualizations necessary to tell the narrative we are trying to make. Although the design and interaction element is a big part of our project. Our visualizations would still be functional without the animations in a scrolley-telling. 

\bibliographystyle{abbrv}
\bibliography{prospectus}
\end{document}
